\documentclass[12pt]{article}

\input 4mbapreamble

%%%%%%%%%%%%%%%%%%%%%%%%%%%%%%%%%%%
%% FANCY HEADER AND FOOTER STUFF %%
%%%%%%%%%%%%%%%%%%%%%%%%%%%%%%%%%%%
\usepackage{fancyhdr,lastpage}
\usepackage[makeroom]{cancel}
\pagestyle{fancy}
\fancyhf{} % clear all header and footer parameters
%%%\lhead{Student Name: \theblank{4cm}}
%%%\chead{}
%%%\rhead{Student Number: \theblank{3cm}}
%%%\lfoot{\small\bfseries\ifnum\thepage<\pageref{LastPage}{CONTINUED\\on next page}\else{LAST PAGE}\fi}
\lfoot{}
\cfoot{{\small\bfseries Page \thepage\ of \pageref{LastPage}}}
\rfoot{}
\renewcommand\headrulewidth{0pt} % Removes funny header line
%%%%%%%%%%%%%%%%%%%%%%%%%%%%%%%%%%%

\begin{document}

\begin{center}
{\bfseries Mathematics 4MB3/6MB3 Mathematical Biology\\
\smallskip
2017 ASSIGNMENT {\color{blue}1}}\\
\medskip
\underline{\emph{Group Name}}: \texttt{{\color{blue}Model Students}}\\
\medskip
\underline{\emph{Group Members}}: {\color{blue}Nicole Dumont, Melody Fong, Connor LeDrew, Carolina Weishaar}
\end{center}

\section{Analysis of the SI model}

The SI model can be written
%
\begin{equation}\label{E:SI}
  \frac{dI}{dt} = \beta I(N - I) \,,
\end{equation}
%
where $I$ denotes prevalence and $N=S+I$ is the total population size.

\begin{enumerate}[(a)]
  \item Prove that the endemic equiliibrium (EE) is a globally asymptotically stable (GAS) equilibrium by finding an appropriate Lyapunov function.  Note that ``global'' here refers to all biologically relevant initial conditions except the (unstable) disease free equilibrium (DFE).  \\
\emph{Hint:} Lyapunov functions often look paraboloidal. \\
\emph{Note:} Notions of stability and Lyapunov functions were discussed in Math 3F03 Lecture 27 in 2013 (\url{http://www.math.mcmaster.ca/earn/3F03}).

{\color{blue}
\begin{proof}
{\color{magenta}

	The EE of ~\eqref{E:SI} is $I_* = N$.
	Let $L$ be the function
	
	\begin{equation}\label{E:L}
 		L(I) = S^2 = (N-I)^2.
	\end{equation}
	
	Since
	
	\begin{align}
		L(I_*) &= 0 \\
		L(I) &> 0 \quad \forall I \neq I_* \\
		\dot{L}(I) &= -2I(N-I)^2 < 0 \quad \forall I \neq I_* ,
	\end{align}
	
	$L$ is a strict Lyapunov function and $I_*$ is a globally asymptotically stable equilibrium.
}
\end{proof}
}

  \item In class we proved only stability of the EE, not asymptotic stability.  Prove GAS ``directly'' in two distinct ways: 
\begin{enumerate}[(i)]
\item find the exact solution of the model and take the limit as $t\to\infty$, and conclude that every solution that starts in the interval $(0,N)$ converges to the EE (this approach works only in situations where you can find the exact solution); 

{\color{blue}
\begin{proof}
{\color{magenta}

	\begin{align}
		\frac{dI}{dt} &= \beta I(N - I) \\
		 \int_{I_0}^{I} \frac{dI'}{I'(N-I')} &= \beta \int_{0}^{t}  dt' \\
		\int_{I_0}^{I} \left ( \frac{1/N}{I'} + \frac{1/N}{N-I'} \right ) dI' &= \beta t \\
		\frac{1}{N} \left ( \ln \left (\frac{I}{I_0} \right ) - \ln \left (\frac{N-I}{N-I_0} \right ) \right ) &= \beta t \\
		\ln \left ( \frac{I(N-I_0)}{I_0(N-I)} \right ) &= \beta Nt \\
		\frac{I}{N-I} &= \frac{I_0}{N-I_0} e^{\beta Nt} \\
		I(t) &= \frac{NI_0e^{\beta Nt}}{N+I_0(e^{\beta Nt} -1)}
	\end{align}

	This is the exact solution of ~\eqref{E:SI}. Taking the limit as $t \rightarrow \infty$ and using L'Hôpital's rule gives

	\begin{align}
		\lim_{t\to\infty} \frac{NI_0e^{\beta Nt}}{N+I_0(e^{\beta Nt} -1)} &= \lim_{t\to\infty} \frac{N^{\cancel 2} \cancel{\beta I_0e^{\beta Nt}}}{\cancel{N \beta I_0e^{\beta Nt}} } \\
													    &= N
	\end{align}
}
\end{proof}
}

\item given $\epsilon>0$, prove that for any $I(0)\in(0,N)\ \exists t<\infty$ such that $I(t)\in[N-\epsilon,N)$ and use this to establish GAS (this approach also works for models that cannot be solved exactly).

{\color{blue}
\begin{proof}
{\color{magenta}
	
	\begin{equation}
\frac{dI}{dt} = \beta I(N-I) = 0 
\end{equation}

$\therefore$ I = 0 and I = N are fixed points. For $\epsilon$ $>$ 0, I = N - $\epsilon$,

\begin{align}
\frac{dI}{dt} &= \beta (N-\epsilon)(N-N+\epsilon) \\
&= \beta (N-\epsilon)(\epsilon) \\
&\geq 0
\end{align}

So, given any $\epsilon$ $>$ 0 and I(t=0)$\in$(0,N), I = N - $\epsilon$ is monotonically increasing, converging to the fixed point I = N $\forall$ t$\in$[0,$\inf$) whenever I$\in$[N-$\epsilon$,N).
}
\end{proof}
}

\end{enumerate}
\end{enumerate}

\section{Analysis of the basic SIR  model}

The basic SIR model is specified by the following system of differential equations.
\begin{subequations}
\begin{align}
\frac{dS}{dt} &= -\R_0 SI \\
\noalign{\vspace{5pt}}
\frac{dI}{dt} &= \R_0 SI - I \\
\noalign{\vspace{5pt}}
\frac{dR}{dt} &= I
\end{align}
\end{subequations}
The state variables $S$, $I$ and $R$ are the proportions of the population that are susceptible, infectious and removed, respectively.  The parameter $\R_0$ is the basic reproduction number.  The time unit has been chosen to be the mean infectious period for convenience.

\begin{enumerate}[(a)]

\item A quantity of some practical importance is the \term{peak prevalence} of disease in the population, \emph{i.e.,} the maximum proportion of the population that is simultaneously infected.  Find an exact expression for the peak prevalence, given initial conditions ($S_0,I_0$).  Why might a public health official want to know this quantity?

{\color{blue}
\begin{proof}[Solution]
{\color{magenta}

	Peak prevalence is an extrema of $I$ so it occurs when 
	\begin{align}
		\frac{dI}{dt} &= I(\R_0 S_p - 1) = 0 \\
		S_p &= \frac{1}{\R_0}
	\end{align}
	Using the equation for the solution curves derived in class, one can find the peak prevalence.
	\begin{align}
		I + S - (I_0+S_0) &= \frac{1}{\R_0}\ln{\frac{S}{S_0}} \\
		I_p + \frac{1}{\R_0} - (I_0+S_0) &= \frac{1}{\R_0}\ln{\frac{1}{\R_0 S_0}} \\
		I_p &= (I_0+S_0) - \frac{1}{\R_0} \left (\ln{\R_0 S_0} +1 \right )
	\end{align}

}
\end{proof}
}

\item It would be helpful to have an analytical expression for the solution of the model.  Most valuable would be a formula for $I(t)$, which is most closely related to time series data.  You probably will not find a formula for $I(t)$ (extra credit if you do!!) but it is definitely possible to find an exact expression that relates $R$ (proportion removed) and $t$ (time).

\begin{enumerate}[(i)]

\item Find such an expression.  \emph{Hint:} Combine the equations for $dS/dt$ and $dR/dt$ into one equation that can be reduced to an integral, and recall that $S+I+R=1$.  \emph{\underline{Note}:} You will end up with an expression for $t$ as a function of $R$, not $R$ as a function $t$.

{\color{blue}
\begin{proof}[Solution]
{\color{magenta}

	\begin{align}
		\frac{dS}{dR} &= \frac{dS}{dt} \frac{dt}{dR} \\
				     &= -\R_0 S \\
		\int_{S_0}^{S} \frac{dS'}{S'} &= -\R_0 \int_{0}^{R} dR' \\
		\ln{\frac{S}{S_0}} &= -\R_0 R \\
		S &= S_0 e^{-\R_0 R} = 1 - R - \frac{dR}{dt} \\
		\frac{dR}{dt} &= 1 - R - S_0 e^{-\R_0 R} \\
		\int_{0}^{t} dt = t(R) &= \int_{0}^{R} \frac{dR'}{1 - R' - S_0 e^{-\R_0 R'}}
	\end{align}
	Since $R_0 R$ is small we can Taylor expand $e^{-\R_0 R'}$ to get
	\begin{align}
		t(R) &= \int_{0}^{R} \frac{dR'}{1 - S_0 + (S_0\R_0 -1)R' -\frac{1}{2}S_0\R_0^2R'^2} \\
		&= \int_{0}^{R} \frac{dR'}{-\frac{1}{2}S_0\R_0^2 \left (R^\prime - \frac{S_0\R_0 - 1}{S_0\R_0^2}  \right )^2  +1-S_0+\frac{(S_0\R_0 -1)^2}	{2S_0\R_0^2}  }
	\end{align}
	
	Let $u=\frac{1}{\sqrt{2}}\sqrt{S_0}\R_0 \left (R' - \frac{S_0\R_0 - 1}{S_0\R_0^2}  \right )$ and $du=\frac{1}{\sqrt{2}}\sqrt{S_0}\R_0 dR'$.
	
	\begin{equation}
		t(R) =\frac{\sqrt{2}}{\sqrt{S_0}\R_0} \frac{2S_0\R_0^2}{S_0\R_0^2(S_0-2)+2S_0\R_0 -1} \int \frac{du}{1+\frac{2S_0\R_0^2}{S_0\R_0^2(S_0-2)+2S_0\R_0 -1}u^2} 
	\end{equation}
	
	Let $v=\frac{\sqrt{2S_0}\R_0}{\sqrt{S_0\R_0^2(S_0-2)+2S_0\R_0 -1}}u$ and $dv=\frac{\sqrt{2S_0}\R_0}{\sqrt{S_0\R_0^2(S_0-2)+2S_0\R_0 -1}}du$.
	
	\begin{align}
		t(R) &= \frac{\sqrt{2}}{\sqrt{S_0\R_0^2(S_0-2)+2S_0\R_0 -1}} \int \frac{dv}{1+v^2}  \\
		&= \frac{\sqrt{2}}{\sqrt{S_0\R_0^2(S_0-2)+2S_0\R_0 -1}} \tan ^{-1} \left ( \frac{S_0\R_0^2R' - S_0\R_0 +1}{\sqrt{S_0\R_0^2(S_0-2)+2S_0\R_0 -1}} \right ) \Bigg |_0^R \\
		&= \frac{\sqrt{2}}{\sqrt{S_0\R_0^2(S_0-2)+2S_0\R_0 -1}} \Bigg ( \tan ^{-1} \left ( \frac{S_0\R_0^2R - S_0\R_0 +1}{\sqrt{S_0\R_0^2(S_0-2)+2S_0\R_0 -1}} \right ) \nonumber \\
		&-\tan ^{-1} \left ( \frac{1 - S_0\R_0 }{\sqrt{S_0\R_0^2(S_0-2)+2S_0\R_0 -1}} \right )  \Bigg ) 
	\end{align}

}
\end{proof}
}

\item Use your expression for $t(R)$ to find an expression for the time at which peak prevalence will occur.  Why might this be useful?

{\color{blue}
\begin{proof}[Solution]
{\color{magenta}

\begin{align}
		R_p &= 1- \frac{2}{\R_0} - (I_0+S_0) - \frac{1}{\R_0} \ln{\R_0 S_0}  \\
		t_p &= \frac{\sqrt{2}}{\sqrt{S_0\R_0^2(S_0-2)+2S_0\R_0 -1}} \nonumber \\
		&  \Bigg ( \tan ^{-1} \Bigg ( \frac{S_0\R_0^2 \left ( 1- \frac{2}{\R_0} - (I_0+S_0) - \frac{1}{\R_0} \ln{\R_0 S_0}  \right ) - S_0\R_0 +1}{\sqrt{S_0\R_0^2(S_0-2)+2S_0\R_0 -1}} \Bigg ) \nonumber \\
		&-\tan ^{-1} \left ( \frac{1 - S_0\R_0 }{\sqrt{S_0\R_0^2(S_0-2)+2S_0\R_0 -1}} \right )  \Bigg ) 
	\end{align}

}
\end{proof}
}

\item How could your expressions be used to compare with the time series for pneumonia and influenza in Philadelphia in 1918?  (Don't actually do it; just clearly explain your thinking including any assumptions you are making.)  Would you advise your assistant who just graduated with a degree in math and biology to do this (to help you prepare your report for the public health agency)?  Why or why not?

{\color{blue}
\begin{proof}[Answers]
{\color{magenta} The data collected on the Influenza epidemic of 1918 in Philadelphia is outlined by the number of deaths per day, within the subset of the population that exhibited symptoms linked to pneumonia or influenza. As a result, the expression derived in the previous question can be used to estimate the value of the parameters, $/gamma$ and $/beta$, by first rearranging the expression in terms of $R$ as a function of $t$ and comparing the curve corresponding to different values of ,$S_0$ and $R_0$, to the mortality curve of the data. The values of $S_0$ and $R_0$ that produces a curve akin to the curve generated by the observed data, corresponds to the final estimate of the mortality curve,$S_0$ and $R_0$. Through the use of this curve and its derivative, the value of $/gamma$ and $/beta$ can be derived, which leads to an approximation of the initial growth rate (assuming $S /approx 0$), the mean infectious period and the basic reproduction number.  However in the SIR model, $R(t)$ represents the number of individuals that have either recovered or died, as a result since the data collected does not contain information about the individuals that recovered from the illness, this data may not be sufficient to make an accurate estimation of $/gamma$ and $/beta$. Therefore, due to the incompleteness of the data I would not recommend a newly graduated student to use this model within a public health report.}
\end{proof}
}

\item Is it possible to find an exact analytical expression for $t$ as a function $S$?
  
{\color{blue}
\begin{proof}[Solution]
{\color{magenta}

	\begin{align}
		\frac{dS}{dt} &= -\R_0 SI \\
		&= -\R_0 S(1-S+\frac{1}{\R_0}\ln{\frac{S}{S_0}}) \\
		\int_{0}^{t} dt = t(S) &= \int_{S_0}^{S} \frac{dS'}{-\R_0 S(1-S+\frac{1}{\R_0}\ln{\frac{S}{S_0}})}
	\end{align}
	This integral cannot be solved in terms of exact analytical expressions. 

}
\end{proof}
}

\end{enumerate}

\item Prove that all solutions of the basic SIR model approach $I=0$ asymptotically, and explain why this makes biological sense.  \emph{Hint:} Is the function $L(S,I)=S+I$ a Lyapunov function?

{\color{blue}
\begin{proof}[Solution]
{\color{magenta}\dots beautifully clear and concise text to be inserted here\dots}
\end{proof}
}

\end{enumerate}


\bigskip

\centerline{\bf--- END OF ASSIGNMENT ---}

\bigskip
Compile time for this document:
\today\ @ \thistime

\end{document}
